\documentclass{article}

\usepackage[francais]{babel}
\usepackage[utf8]{inputenc}
\usepackage{hyperref}

\begin{document}

\title{QF-Test - Tutoriel français}
\maketitle
\newpage

\tableofcontents
\newpage

\section{Introduction}
\par{
Le but de ce document est de donne un introduction pour faciliter la mise en route en utilisent le logiciel
QF-Test~\footnote{\url{https://www.qfs.de/en/qf-test/download.html}}. En particulier, l'application pratique
du logiciel sera discutée. Dans ce sens ce tutoriel dispense largement des détails techniques et ce concentre
sur les connaissances importantes pour la création rapide de tests.
}
\par{
En aucun cas, ce tutoriel ne devrait être considéré comme complet, car de nombreux détails techniques ont été
délibérément omis. Si celles vous intéressez également, alors jetez un coup d'oeil dans le
manuel~\footnote{\url{https://www.qfs.de/en/qf-test-manual.html}}. De plus, je voudrais rappeler que QFS
offre des formations. Si ces formations vous intéressez, consultez
la page d'entraînement~\footnote{\url{http://www.qfs.de/en/qftest/training.html}}.
}
\newpage

\section{Installation de QF-Test}
\par{
Commençons avec l'installation de QF-Test. En principe, QF-Test fonctionne sur Windows, Linux et macOS.
Pour connaître les exigences d'installation de la version de QF-Test actuelle, il vaut la peine de jeter un
coup d'oeil dans le chapitre~1.1.1 du
manuel~\footnote{\url{https://www.qfs.de/en/qf-test-manual/lc/manual-en-user\_installation.html\#sec_N120}}.
}
\par{
Il y en a des options nombreuses de configuration pour l'installation de QF-Test. Par exemple, c'est
possible d'installer QF-Test sur une unité de réseau ou d'enregistrer certains fichiers dans d'autres
dossiers.
}
\par{
Ici, cependant, seule l'installation standard sera discutée. Si vous êtes interessé par d'autres types
d'installation je vous conseille de lire le
chapitre~1.1~\footnote{\url{https://www.qfs.de/en/qf-test-manual/lc/manual-en-user_installation.html}}
et~25~\footnote{\url{https://www.qfs.de/en/qf-test-manual/lc/manual-en-bp_howtostart.html}} dans le manuel.
\par{
Les chapitres suivants traitent l'installation de QF-Test sur les systèmes d'exploitation différents.
}

\subsection{Installation sur Linux}
\par{
Pour installer QF-Test sur Linux, il faut d'abord choisir le dossier d'installation. Dossiers habituels
sont par example /opt, /usr/local ou le répertoire HOME de l'utilisateur. Assurez-vous d'avoir le droit de
lecture et d'écriture pour le dossier de votre choix. Si vous voulez installer une nouvelle version de
QF-Test reutilisez le même dossier.
}
\par{
Changez dans ce dossier et téléchargez l'archive au format .tar.gz de la
site web~\footnote{\url{https://www.qfs.de/en/qf-test/download.html}}. Après la décrompression de
l'archive, vous trouvez les fichier setup.sh/setup.ksh sous qftest/qftest-\$(version)/. Pour finaliser
l'installation de QF-Test tournez setup.sh si vous utilisez un
bsh~\footnote{\url{https://fr.wikipedia.org/wiki/Bourne_shell}}. Pour
ksh~\footnote{\url{https://fr.wikipedia.org/wiki/Korn_shell}} faitez tourner setup.ksh.
}
\par{
Après vous pouvez démarrer QF-Test en executent l'application qftest qui ce trouve dans le dossier
qftest/qftest-\$(version)/bin.
}

\subsection{Installation sur macOS}
\par{
C'est possible d'installer QF-Test sur macOS en suivant le sous-chapitre "l'installation sur Linux".
Néanmoins, sur la site web~\footnote{\url{https://www.qfs.de/en/qf-test/download.html}} vous trouvez
également une image-disque (.dmg, façon d'installation recommandée).
}
\par{
Après l'activation de l'image-disque, copiez l'application QF-Test dans le répertoire des programmes.
En suite, vous pouvez exécuter QF-Test de ce répertoire.
}

\subsection{Installation sur Windows}
\par{
Après d'avoir télécharger le programme d'installation de Windows sur la site
web~\footnote{\url{https://www.qfs.de/en/qf-test/download.html}} (le fichier .exe) démarrez ce programme.
Le programme d'installation de Windows consiste en des étapes habituelles. C'est à dire après la confirmation
des termes de la licence, vous pouvez choisir certain paramètres comme le dossier d'installation.
}
\par {
À la fin du programme, l'installateur vous offrez la posibilité de démarrer qftest.
}

\subsection{Importation de la licence}
\par{
Quand vous démarrez QF-Test pour la première fois, QF-Test vous demande d'importer une license. Sans license,
QF-Test tourne seulement en mode démo. Ce mode est un peu limité - par exemple, QF-Test ne vous laissez pas
enregistrer votre travail.
}
\par{
Pour essayer QF-Test demandez une license d'évaluation~\footnote{\url{https://www.qfs.de/en/qf-test/free-trial.html}}.
}
\newpage

\section{Les premières étapes}
\par{

}

\end{document}
