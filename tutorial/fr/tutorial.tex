\documentclass{article}

\usepackage[francais]{babel}
\usepackage[utf8]{inputenc}
\usepackage{hyperref}

\begin{document}
\title{QF-Test - Tutoriel français}
\maketitle

\section{Introduction}
\par{
Le but de ce document est de donne un introduction pour faciliter la mise en route en utilisent le logiciel
QF-Test~\footnote{\url{https://www.qfs.de/en/qf-test/download.html}}. En particulier, l'application pratique
du logiciel sera discutée. Dans ce sens ce tutoriel dispense largement des détails techniques et ce concentre
sur les connaissances importantes pour la création rapide de tests.
}
\par{
En aucun cas, ce tutoriel ne devrait être considéré comme complet, car de nombreux détails techniques ont été
délibérément omis. Si celles vous intéressez également, alors jetez un coup d'oeil dans le
manuel~\footnote{\url{https://www.qfs.de/en/qf-test-manual.html}}. De plus, je voudrais rappeler que QFS
offre des formations. Si ces formations vous intéressez, consultez
la page d'entraînement~\footnote{\url{http://www.qfs.de/en/qftest/training.html}}.
}

\section{Installation de QF-Test}
\par{
Commençons avec l'installation de QF-Test. QF-Test fonctionne sûr Windows, Linux et macOS. Les chapitres
suivants traitent l'installation de QF-Test sûr les systèmes d'exploitation différents.
}
\par{
Sauf si 
}

\section{L'installation de la licence}
\par{
}

\end{document}
