\documentclass{article}

\usepackage[francais]{babel}
\usepackage[utf8]{inputenc}
\usepackage{hyperref}

\begin{document}

\title{QF-Test - Tutoriel français}
\maketitle
\newpage

\tableofcontents
\newpage

\section{Introduction}
\par{
Le but de ce document est de donne un introduction pour faciliter la mise en route en utilisent le logiciel
QF-Test~\footnote{\url{https://www.qfs.de/en/qf-test/download.html}}. En particulier, l'application pratique
du logiciel sera discutée. Dans ce sens ce tutoriel dispense largement des détails techniques et ce concentre
sur les connaissances importantes pour la création rapide de tests.
}
\par{
En aucun cas, ce tutoriel ne devrait être considéré comme complet, car de nombreux détails techniques ont été
délibérément omis. Si celles vous intéressez également, alors jetez un coup d'oeil dans le
manuel~\footnote{\url{https://www.qfs.de/en/qf-test-manual.html}}. De plus, je voudrais rappeler que QFS
offre des formations. Si ces formations vous intéressez, consultez
la page d'entraînement~\footnote{\url{http://www.qfs.de/en/qftest/training.html}}.
}
\newpage

\section{Installation de QF-Test}
\par{
Commençons avec l'installation de QF-Test. En principe, QF-Test fonctionne sûr Windows, Linux et macOS.
Pour connaître les exigences d'installation de la version de QF-Test actuelle, il vaut la peine de jeter un
coup d'oeil dans le chapitre~1.1.1 du
manuel~\footnote{\url{https://www.qfs.de/en/qf-test-manual/lc/manual-en-user\_installation.html\#sec_N120}}.
}
\par{
Il y en a des options nombreuses de configuration pour l'installation de QF-Test. Par exemple, c'est
possible d'installer QF-Test sur une unité de réseau ou d'enregistrer certains fichiers dans d'autres
dossiers.
}
\par{
Ici, cependant, seule l'installation standard sera discutée. Si vous êtes interessé par d'autres types
d'installation je vous conseille de lire le
chapitre~1.1~\footnote{\url{https://www.qfs.de/en/qf-test-manual/lc/manual-en-user_installation.html}}
et~25~\footnote{\url{https://www.qfs.de/en/qf-test-manual/lc/manual-en-bp_howtostart.html}} dans le manuel.
\par{
Les chapitres suivants traitent l'installation de QF-Test sûr les systèmes d'exploitation différents.
}

\subsection{Installation sûr Linux}
\par{
Pour installer QF-Test sûr Linux, il faut d'abord choisir le dossier d'installation. Dans ce dossier
qftest-4.1.6.tar.gz
}
\par{
Sauf si 
}

\subsection{Installation sûr macOS}
\par{
}

\subsection{Installation sûr Windows}
\par{
}

\subsection{Importation de la licence}
\par{
}

\subsection{}
\par{
}

\end{document}
